\section{Conclusions}
\label{sec:conclusions}


% test results 최대 0.787, MS-COCO와 비교했을 때 분류해야하는 클래스의 차이는 있지만 정의된 임무에 대해서 높은 성능이라고 할 수 있다.
In this paper, we have proposed an algorithm that utilizes transfer learning with YOLOv8 for the detecting of driving vehicles on the road and their brake light status.
Our proposed approach offers novelty in three main aspects.
First, we have constructed and publicly released a dataset specifically designed for detecting driving vehicles and brake light status on the road.
Acquiring high-quality datasets remains a challenging task, and by making our dataset and annotation results accessible to everyone, we have provided a foundation for related research.
Second, we have proposed a one-stage brake light detection network that ensures both high accuracy and fast inference speed.
This network, trained with dataset using YOLOv8, takes a single forward image of a driving vehicle as input and detects all driving vehicles in the image while classifying their brake light status as off or on.
Through training and evaluation of models of various sizes, we have achieved high accuracy with a maximum mAP50 of $0.793$.
We have also provided detailed analysis considering various driving environments, including different ambient illumination conditions.
Lastly, we have validated the real-time capability of the proposed network by examining the inference time of all trained models on Nvidia Jetson Nano devices installed in the driving vehicle.
By comparing the trade-off between detection accuracy and inference time, we have obtained a fast inference speed of $133.30$ ms along with a detection performance of mAP50 $0.766$.

Through this research, we aim to make autonomous driving technology safer, more interpretable, and more comfortable by proposing a fast and accurate brake light status detection network.
In future work, we plan to conduct experimental research on the utilization of brake light status detection results for improving the safety, interpretability, and alleviation of motion sickness in autonomous driving systems.
