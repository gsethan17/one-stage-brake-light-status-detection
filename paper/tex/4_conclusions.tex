\section{Conclusions}
\label{sec:conclusions}


In this paper, we have proposed an algorithm that utilizes transfer learning with YOLOv8 for the detecting of driving vehicles on the road and their brake light status.
Our proposed approach offers novelty in three main aspects.
First, we have constructed and publicly released a dataset specifically designed for detecting driving vehicles and brake light status on the road.
Acquiring high-quality datasets remains a challenging task, and by making our dataset and annotation results accessible to everyone, we have provided a foundation for related research.
Second, we have proposed a one-stage brake light detection network that ensures both high accuracy and fast inference speed.
This network, trained with dataset using YOLOv8, takes a single forward image of a driving vehicle as input and detects all driving vehicles in the image while classifying their brake light status as off or on.
Through training and evaluation of models of various sizes, we have achieved high accuracy with a maximum mAP50 of $0.793$.
We have also provided detailed analysis considering various driving environments, including different ambient illumination conditions.
Lastly, we have validated the real-time capability of the proposed network by examining the inference time of all trained models on Nvidia Jetson Nano devices installed in the driving vehicle.
By comparing the trade-off between detection accuracy and inference time, we have obtained a fast inference speed of $133.30$ ms along with a detection performance of mAP50 $0.766$.

We plan to continue our future research to propose more notable brake light status detection algorithm.
One aspect of our future work involves improving the network architecture and considering sequential image inputs for more accurate and faster detection.
In this study, we conducted transfer learning by keeping the YOLOv8 network structure unchanged while changing only the task for driving vehicle and brake light status detection.
Therefore, we will aim to refine the network architecture to be more suitable for the task.
In this study, since we utilized the input data shape of YOLOv8 without any change, it limited us to using only the current single image as the input. 
As mentioned in Section \ref{sec:method_dataset}, it is true that preceding or succeeding images can provide valuable information for the brake light status detection.
Hence, we plan to improve detection performance in future research by considering additional input images, such as preceding frames, along with the current image.
The other aspect of our future work will involve conducting experimental research on the utilization of brake light status detection results for improving the safety, interpretability, and alleviation of motion sickness in autonomous driving systems.
One of the objectives of proposing a fast and accurate brake light status detection in this study is to enhance the safety, interpretability, and comfort of autonomous driving technology.
In conclusion, we plan to conduct empirical studies applying the brake light status detection algorithm to autonomous driving systems.