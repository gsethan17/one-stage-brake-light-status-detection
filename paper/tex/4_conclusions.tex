\section{Conclusions}
\label{sec:conclusions}


% test results 최대 0.787, MS-COCO와 비교했을 때 분류해야하는 클래스의 차이는 있지만 정의된 임무에 대해서 높은 성능이라고 할 수 있다.
In this paper, we proposed an algorithm that utilizes transfer learning with YOLOv8 to detect driving vehicles on the road and their brake light status.
Our proposed approach exhibits novelty in three main aspects.
Firstly, we constructed and publicly released a dataset specifically designed for detecting driving vehicle and brake light status on the road.
While data based training has become relatively easier with advancements in computer technology, acquiring high-quality datasets remains a challenging.
By making the results of all the tasks, including over 16 hours of real-road driving, manual annotation, and preprocessing considering data characteristics and detection objectives, accessible to everyone, we established a foundation for related research.
Secondly, we proposed a one-stage brake light detection network that guarantees both high accuracy and fast inference speed.
The proposed detection network is the results of transfer learning using the introduced dataset with YOLOv8.
This network takes input of a single frontal image of a driving vehicle and detects all driving vehicles in the image, as well as classifies their brake light status as either off or on.
We trained and evaluated models of various sizes and achieved a high accuracy with a maximum mAP50 of $0.793$.
Additionally, we provide detailed analysis results considering various driving environments, such as various in ambient illumination conditions.
Lastly, we validated the real-time capability by examining the inference time of all trained models on Nvidia Jetson Nano devices installed on the driving vehicle.
By comparing the trade-off performance between detection accuracy and inference time, we obtained a fast inference speed of $133.30$ ms along with a detection performance of mAP50 $0.766$.

Through this research, we have proposed the fast and accurate brake light status detection network to make autonomous driving technology safer, more interpretable, and more comfortable.
However, when examining previous research, it is difficult to find empirical studies other than those suggesting that the brake light status detection results improve the ride comfort of ACC \cite{pirhonen2022predictive}.
Therefore, based on the results of this study, we plan to conduct experimental research on the utilization of brake light status detection results for improving the safety, interpretability, and alleviation of motion sickness in autonomous driving systems.
